\documentclass[UTF8,a4paper,14pt]{ctexart}
\usepackage[utf8]{inputenc}
\usepackage{amsmath}
\usepackage{amssymb}
\usepackage{amsfonts}
%for 字體
%https://tug.org/FontCatalogue/
% \usepackage[T1]{fontenc}
% \usepackage{tgbonum}
% \usepackage[bitstream-charter]{mathdesign}
% \usepackage[T1]{fontenc}
% \usepackage{bm}#粗體
%\usepackage{boondox-calo}
\usepackage{textcomp}
\usepackage{fancyhdr}%导入fancyhdrf包
\usepackage{ctex}%导入ctex包
\usepackage{enumitem} %for在Latex使用條列式清單
\usepackage{varwidth}
\usepackage{soul} %for \ul
\usepackage{comment}%\begin{comment}\end{comment}
\usepackage{cancel}%\cancel{}
%\usepackage{unicode-math}

\usepackage[dvipsnames, svgnames, x_11names]{xcolor}

\usepackage[low-sup]{subdepth}
\usepackage{subdepth}

\newcommand{\indep}{\Perp \!\!\! \Perp}

\usepackage{amsthm}
\DeclareMathOperator{\E}{\mathbb{E}}
\DeclareMathOperator{\Var}{\textbf{Var}}
\DeclareMathOperator{\Cov}{\textbf{Cov}}
\DeclareMathOperator{\Cor}{\textbf{Cor}}
\DeclareMathOperator{\X}{\mathbf{X}}
\DeclareMathOperator{\Pro}{\mathbf{P}}
\DeclareMathOperator{\M}{\mathbf{M}}
\DeclareMathOperator{\Id}{\mathbf{I}}
\DeclareMathOperator{\Y}{\mathbf{Y}}
\DeclareMathOperator{\MSFE}{\mathbf{MSFE}}
\DeclareMathOperator{\e}{\mathbb{e}}
\DeclareMathOperator{\V}{\mathbf{V}} 
\DeclareMathOperator{\tr}{\text{tr}}
\DeclareMathOperator{\A}{\textbf{A}}
\DeclareMathOperator{\diag}{diag}
\DeclareRobustCommand{\rchi}{{\mathpalette\irchi\relax}}
\newcommand{\irchi}[2]{\raisebox{\depth}{$#1\chi$}} % inner command, used by \rchi

\DeclareMathSizes{10}{10}{7}{3}

\usepackage[a4paper, margin=1in]{geometry}
% \setlength\parskip{5ex}% it would be better define distance in ex (5ex) 
                         %  or in pt, pc, mm, etc (see edit below)


\usepackage{array, makecell} %


%中英文設定
%\usepackage{fontspec}
% \setmainfont{TeX Gyre Termes}
% \usepackage{xeCJK} %引用中文字的指令集
% %\setCJKmainfont{PMingLiU}
\setCJKmainfont{DFKai-SB}
% \setmainfont{Times New Roman}
% \setCJKmonofont{DFKai-SB}
\pagenumbering{arabic}%设置页码格式
\pagestyle{fancy}
\fancyhead{} % 初始化页眉
\usepackage{advdate}

% \newcommand{\yesterday}{{\AdvanceDate[-1]\today}}

\fancyhead[C]{Stochastic Caculus\quad  R10A21126\quad  WANG YIFAN\quad   \today}
%\fancyhead[LE]{\textsl{\rightmark}}
%\fancyfoot{} % 初始化页脚
%\fancyfoot[LO]{奇数页左页脚}
%\fancyfoot[LE]{偶数页左页脚}
%\fancyfoot[RO]{奇数页右页脚}
%\fancyfoot[RE]{偶数页右页脚}

% \title{{Econometrics HW 05}}
% \author{R10A21126}
% \date{\today}

%\fancyhf{}
\usepackage{lastpage}
\cfoot{Page \thepage \hspace{1pt} of\, \pageref{LastPage}}

\renewcommand{\headrulewidth}{0.1pt}%分隔线宽度4磅
%\renewcommand{\footrulewidth}{4pt}

\allowdisplaybreaks
\usepackage[english]{babel}
%\usepackage{amsthm}
\newtheorem{theorem}{Theorem}[section]
\newtheorem{corollary}{Corollary}[theorem]
\newtheorem{lemma}[theorem]{Lemma}


\usepackage[most]{tcolorbox}

\definecolor{babyblue}{rgb}{0.54, 0.81, 0.94}

\newtcolorbox[auto counter]{mybox}[1]{
    enhanced,
    arc= 1 mm,boxrule=1.5pt,
    colframe=babyblue!80!pink,
    colback=white,
    coltitle=black,
    % colback=blue!5!white,
    attach boxed title to top left=
    {xshift=1.5em,yshift=-\tcboxedtitleheight/2},
    boxed title style={size=small,
    % frame hidden,
    colback=White},
    top=0.15in,
    % fonttitle=\bfseries,
    title= {#1},
    breakable
  }

\newtcolorbox[auto counter]{Problem}[2][]{
    enhanced,drop shadow={Pink!50!white},
    colframe=pink!80!white,
    fonttitle=\bfseries,
    title=Problem ~\thetcbcounter. #2,
    %separator sign={.},
    coltitle=black,
    colback=pink!15,
    top=0.15in,
    breakable
  }

\newenvironment{solution}
  {\renewcommand\qedsymbol{$\blacksquare$}\begin{proof}[Solution]}
  {\end{proof}}

\theoremstyle{definition}
\newtheorem{definition}{Definition}[section]

%\theoremstyle{notation}
\newtheorem*{notation}{\underline{Notation}}
%\newtheorem*{convention}{\underline{Convention}}
\newtheorem*{convention}{\underline{Convention}}

\theoremstyle{remark}
\newtheorem*{remark}{Remark}

\newenvironment{amatrix}[2]{%% [2] for 2 parameters 
  \left[\begin{array}
    %{cc\,|\,cc}
    %  {@{}*{#2}{c}\,|\,c*{#1}{c}}
     {{}*{#1}{c}\,|\,c*{#2}{c}}
}{%
  \end{array}\right]
}
% For augmented matrix  
%https://tex.stackexchange.com/questions/2233/whats-the-best-way-make-an-augmented-coefficient-matrix


\begin{document}

\section{開始講故事}

賭博自古就有,西方,宗教革命,就從god's will變成當作一門學問來看。

學了機率論,慢慢會改變一些直覺。

De Moivre 在1733第一個證明中央極限定理

牛頓是鑄幣局長,在生計上幫他

老師1983在MIT訪問,大家在慶祝CIT證明250周年

20世紀才開始,幾率論有公理化的探討。

Kolmogorov 1933

做證明,如果把題目搞得很清楚,大概就做好了一半了。

社科的問題,用數學模型,如何描述

模型一定是不對的,只是對理想狀態的描述,要思考Uncertainty的問題

\section{開始講一些基本觀念}
\subsection{$\sigma$ - algebra}
你們知道Sigma-algebra對不對

^^

\begin{equation}
  \sigma - algebra
\end{equation}

random phenomenon

\begin{equation}
  \begin{aligned}
    \X: \Omega \rightarrow \mathbb{R} \\
    A=\left\{\X\leq a\right\} \\
    B=\left\{\X\geq b\right\}    
  \end{aligned}
\end{equation}

\begin{equation}
  \begin{aligned}
    \mathfrak{B} :\text{ Borel $\sigma$- algebra
    the smallest  $\sigma$ -algebra
    containing all the open sets (interxxx) in $\mathbb{R}$}
  \end{aligned}
\end{equation}

\begin{equation}
  \begin{aligned}
   \Omega \mathcal{F}  \overset{X}{\rightarrow }\Omega' \mathcal{F}' \\
x^{-1}(\mathcal{F} ') \subset \mathcal{F}    
  \end{aligned}
\end{equation}
$X$ is called a $\Omega$- xxx

random variable (function, element)

\[\sigma (x)\equiv X^{-1}(\varepsilon )\]

如果沒有$\sigma$ - algebra的基礎,很難繼續

\subsection{Probability}
A Probability P on $\mathcal{F} $
\[\mathsf{P} :\mathcal{F} \rightarrow [0,1]\]

(1) P($\Omega$)=1

(2)P($\cup_1^{\infty} A_n$ ) =$ \sum_{n = 1}^{\infty} A_n   $

if \(A_n\in \mathcal{F}  \) and \(A_n \bigcap A_M = \emptyset  \) if \(n\neq m\)

(3) Countable additivity

finite additivity.

vector measure:

\[\mathsf{P} :\mathcal{F} \rightarrow \mathcal{B} \]
P($\cup_1^{\infty} A_n$ ) =$ \sum_{n = 1}^{\infty} P(A_n )  $


\subsection{Stochastic Process 隨機過程}

現在都叫Stochastic Process,因爲Stochastic看起來比較有學問,以前叫作random.

起名字很重要。

有人説random number叫亂數翻譯很不好。

\[I={1}\]
\[\left\{X(t);t\in I\right\}\]

I: index set

\(X(t)\) is a Stochastic process if \(X(t)\) is a r.v. \(\forall t \in I\)

\(I = [0,T)\) or \(I=[0,\infty)\)

\(I=\mathbb{R} \) or \(I=\mathbb{R}^k \) random field

I 有次序

\[(\Omega, \mathcal{F} ,\left\{\mathcal{F} _t\right\}) t\in I\]
Filtration
\[\forall t , \mathcal{F}_t \subseteq \mathcal{F}, \mathcal{F}_s \subseteq \mathcal{F}, if s\leq t\]
\( \mathcal{F}_t\): information w.r.t time t


\[\mathcal{F}_{t^{+}}\equiv \underset{s>t}{\bigcap}\mathcal{F}_s \]
right continuous Filtration
\[\mathcal{F}_t = \mathcal{F}_{t+}\]
通常都假設右連續

---

\(X(t)\) is a Stochastic process adapted to \(\{\mathcal{F}_t\}\) if \(X(t)\in\mathcal{F}_t, \forall t\)

\(\sigma(X(t))\subseteq \mathcal{F}_t \)

Usual condition 

\(\Omega, \mathcal{F}, \{\mathcal{F}_t\},P\)

\(\mathcal{F}\)先做condition under P

if \(\mathcal{F}_t\) contains all the null sets in \(\mathcal{F}\) and \(\{\mathcal{F}_t\}\) is right continuous

*

統計和機率的差異

統計比較像自然科學, induction

機率比較, deduction

X:

什麽叫做我知道X?

數學,至少要知道X的distribution

\(F_X(t) = P(X \leq t)\,  \forall t\)

X and Y 什麽叫2個隨機變數一樣?

\(F_X(t) =F_Y(t) \, \forall t\)

\(X\overset{w}{\sim}Y\) weakly

\(X\overset{d}{\sim}Y\)


X and Y ,什麽叫2個隨機過程一樣?

最弱的要求:
finite-dimensional distributions 要一樣
\[P_{t_1,\ldots,t_n}(x_1,\ldots,x_n) = P(X(t_1)\leq x_1,\ldots,X(t_n)\leq x_n)\]

\[X(t),t\in I\]
\[P_{t_1,\ldots,t_n}(x_1,\ldots,x_n) = P(X(t_1)\leq x_1,\ldots,X(t_n)\leq x_n), n\in\mathbf{N} ,t_1,\ldots,t_n \in [0,\infty)\]

the family of all finite-dim distributions is consistent
\[P_{t_1,\ldots,t_n}(x_1,\ldots,x_n) = P_{t_{\pi(1)},\ldots,t_{\pi(n)}}(X_{\pi(1)},\ldots,X_{\pi(n)})\]
\[\pi:\{1,\ldots,n\}\rightarrow \{1,\ldots,n\} \]
permutation


\subsection{隨機過程的不同解釋}
1. \(\{X(t),t\leq s\},X(t)\) is a r.v.


2. \(L^2\) Theory , \(\E X^2(t) <\infty \forall t\)

\(t \mapsto X(t,\cdot )\in L^2(P)=\{Y:\E y^2 <\infty\} \)
is a curve in \(L^2(P)\)

通常都假設
\(X(t) \) is continuous in \(L^2(P) \)

spectral Theory

Karhunen–Loève expansion

北歐

\(cov(t,s) = \E X(t)\)

\end{document}